\documentclass{ctexart}
\usepackage{amsmath}
\usepackage[hmargin=1.25in,vmargin=1in]{geometry}

\begin{document}

\pagestyle{empty}

19.\quad(17分)复数的引入在数学发展史上具有重要的意义。

(1)已知对任意函数$f(x)$,有$f(x)=\sum_{n = 0}^{\infty} \frac{f^{(n)}(x)}{n!}x^n$,其中$f^{(n)}(x)$表示$f(x)$的$n$阶导函数。
$n!=1\times 2\times \cdots \times n$。

试证:$e^{\rm{i}\theta}=\rm{cos}(\theta)+\rm{isin}(\theta)$;(5分)

~\\

(2)定义新运算“$\oplus $”和“$\oplus _a^b$”,如果$F(x)$的导函数为$f(x)$,那么$F(x)=\oplus f(x)$;$\oplus _a^bf(x)=F(b)-F(a)$。

对于自变量为实数的函数,我们可以将其自变量替换为复数。已知对有些函数,我们可以对替换后的函数做$\oplus _a^b$运算,其中$a$、$b$为复数,且
满足$\oplus _a^bf(z)+\oplus _b^cf(z)=\oplus _a^cf(z)$。我们称这样的函数为“好函数”。我们接下来研究的函数都为“好函数”。
利用$\oplus _a^bf(z)+\oplus _b^cf(z)=\oplus _a^cf(z)$,我们可以将$\oplus _a^af(z)$分解为沿一个围道进行多次操作的结果。若无论如何
选取围道,$\oplus _a^af(z)$恒等于0,则称$f(z)$为“纯函数”。否则称$f(z)$为“非纯函数”。

试证:若非纯函数$f(z)$满足$f(z)=\sum_{n = -\infty}^{\infty} c_n(z-z_0)^n$,则若计算$\oplus _a^af(z)$时所选取的围道中仅包含$z_0$一个使$f(z)$无意义的点,
则$\oplus _a^af(z)=2\pi \rm{i}c_{-1}$;(8分)

~\\

(3)已知若计算$\oplus _a^af(z)$时所选取的围道中包含多个无意义的点,则其值等于所有计算时选取的围道仅包含一个不同的无意义点所得$\oplus _a^af(z)$的和。

试求:$\oplus _0^{2\pi}\frac{1}{1+\rm{sin}^2\it{(x)}}$;(4分)

~\\
~\\
~\\

个人感觉算是把目前流行的出题的要素全糅合进去了,基本包括:较高的数学背景;有现成的符号不用,偏好定义新运算;偏好给已经有名称的东西命名;上一问所证为求解下一问的引理;
最后一问思维跨度大。

放过孩子们吧。

\hfill\it{by}\rm{食司}

\hfill\zhtoday








\end{document}